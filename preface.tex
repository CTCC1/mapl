\chapter*{Об авторе\markright{Об авторе}{}}
\addcontentsline{toc}{chapter}{Об авторе}

Магнус Лай Хетланд "--- опытный программист на Python, использующий этот язык с 90-х годов. Также он читает курс теории алгоритмов в Норвежском университете науки и технологий и изучает алгоритмы почти десять лет. Хетланд "--- автор книги <<Beginning Python>>.

\chapter*{О техническом рецензенте\markright{О техническом рецензенте}{}}
\addcontentsline{toc}{chapter}{О техническом рецензенте}

Алекс Мартелли родился и вырос в Италии и имеет степень доктора электротехники Болонского университета. Он написал книгу <<Python in a nutshell>> и был соавтором книги <<Python Cookbook>>. Является членом PSF, в 2002 получил премию Activators’ Choice Award, а в 2006 "--- премию Frank Willison Award за вклад в развитие сообщества Python. Сейчас живет в Калифорнии и работает ведущим разработчиком в Google. Больше о нем можно узнать по адресу: www.google.com/profiles/aleaxit; биографию можно найти на http://en.wikipedia.org/wiki/Alex\_Martelli.


\chapter*{Благодарности\markright{Благодарности}{}}
\addcontentsline{toc}{chapter}{Благодарности}

Спасибо всем, кто так или иначе внес свой вклад в написание этой книги. Конечно же, это мои наставники по теории алгоритмов, Арне Халаас и Бьорн Олстад, вся команда Apress и мой великолепный технический рецензент Алекс. Спасибо Нильсу Гримсмо, Йону Мариусу Венстаду, Оле Эдсбергу, Рольву Сеехуусу и Йоргу Родсо за полезные замечания. Спасибо моим родителям, Керсти Лай и Тору М. Хетланду, и моей сестре Анне Лай-Хетланд за их интерес и поддержку, а также моему дяде Акселю за проверку моего французского. И в заключение огромная благодарность Python Software Foundation за разрешение использовать части стандартной библиотеки Python и Рэндаллу Манро за разрешение включить в книгу несколько его чудесных комиксов XKCD.


\chapter*{Предисловие\markright{Предисловие}{}}
\addcontentsline{toc}{chapter}{Предисловие}

Эта книга родилась как слияние трех моих увлечений: алгоритмов, программирования на Python и объяснения людям разных вещей. Мне кажется, что эстетичность подразумевает три действия: вы находите правильный способ сделать что-либо, приглядываетесь к нему до тех пор, пока не проявится намек на элегантность, а затем шлифуете его до блеска. Ну, или хотя бы делаете чуть красивее. Конечно, когда требуется выполнить очень много работы, вы, вероятно, не сможете довести свои решения до совершенства. К счастью, большая часть материалов в этой книге близка к совершенству, так как я описываю действительно прекрасные алгоритмы и доказательства и использую один из лучших языков программирования. Что же до третьей составляющей, я серьезно поработал, чтобы объяснить вещи так просто, как это только возможно. Но даже при этом, я уверен, что во многих местах этого не достиг, так что если у вас есть советы по улучшению книги, я буду рад их услышать. Кто знает, может быть какие-то из ваших идей могут войти в следующее издание? Сейчас же я надеюсь, что эта книга вам понравится и вдохновит вас двигаться дальше. Если вы можете, используйте ее, чтобы сделать мир чуточку лучше.
%%% Local Variables: 
%%% mode: latex
%%% TeX-master: "mapl"
%%% End: 
